% Specify the type of document
\documentclass[12pt]{article}

% Load a number of useful packages
\usepackage{graphicx}
\usepackage{amsmath,amssymb,amsfonts,amsthm}
 \usepackage[margin=1.0in]{geometry}
\usepackage[colorlinks=true]{hyperref}
\usepackage{cite}
\usepackage[caption=false,font=footnotesize]{subfig}

% Two more packages that make it easy to show MATLAB code
\usepackage[T1]{fontenc}
\usepackage[framed,numbered]{matlab-prettifier}
\lstset{
	style = Matlab-editor,
	basicstyle=\mlttfamily\small,
}

% Say where pictures (if any) will be placed
\graphicspath{{./pictures/}}

% Define title, author, and date
\title{AE353: Design Problem 01}
\author{J.Clements}

% Start of document
\begin{document}
    
% Put the title, author, and date at top of first page
\maketitle


\section{Goal}

The code provided in \lstinline!DesignProblem01! simulates the rotational motion of a spacecraft. This spacecraft has actuators that can apply torque about two different axes. This spacecraft also has sensors---an inertial measurement unit (IMU)---to measure its angular velocity. The spacecraft starts with some random angular velocity. The goal is to achieve an angular velocity of $w = $
$M =
    \begin{bmatrix}
    1 & 0 & 0
    \end{bmatrix}$ rad/s.

\section{Model}

The rotational motion of the spacecraft, if it is modeled as a single rigid body, is governed by the ordinary differential equations
\begin{equation}
    \tau_{1} = J_{1}\dot{w}_{1}-(J_{2}-J_{3})w_{2}w_{3}
\end{equation}
\begin{equation}
    0 = J_{2}\dot{w}_{2}-(J_{3}-J_{1})w_{3}w_{1} 
\end{equation}
\begin{equation}
    \tau_{3} = J_{3}\dot{w}_{3}-(J_{1}-J_{2})w_{1}w_{2},
\end{equation}
where $w_{1},w_{2},w_{3}$ are the components of angular velocity, $J_{1},J_{2},J_{3}$ are the principle moments of inertia, and $\tau_{1},\tau_{3}$ are the two different torques that can be applied.

\section{State Space Model Linearization}

In order to determine the linear model, we first define the State, $x$, Input, $u$, and Output, $y$.  Our goal is to simplify our equations of motion to the following:
\begin{align*}
\dot{x} = Ax + Bu \\ y = Cx + Du
\end{align*}

To begin, we define the state, input, and output as the following:\\
\begin{align*}
x = 
    \begin{bmatrix}
    \omega _{1} \\ \omega _{2} \\ \omega _{3}
    \end{bmatrix},\enspace
u = 
    \begin{bmatrix}
    \tau _{1} \\ 0 \\ \tau _{3}
    \end{bmatrix},\enspace
y = 
    \begin{bmatrix}
    \omega _{1} \\ \omega _{2} \\ \omega _{3}
    \end{bmatrix}
\end{align*}    

Gathering all this information, we move to MatLab to utilize Jacobian() in order to find A, B, C, and D for the state space model above.

\begin{align*}
A = 
    \begin{bmatrix}
    0 & (\omega _{3} \times \frac{J _{2} - J _{3}}{J _{1}}) & (\omega _{2} \times \frac{J _{2} - J _{3}}{J _{1}}) \\ -(\omega _{3} \times \frac{J _{1} - J _{3}}{J _{2}}) & 0 & -(\omega _{1} \times \frac{J _{1} - J _{3}}{J _{2}}) \\ (\omega _{2} \times \frac{J _{1} - J _{2}}{J _{3}}) & (\omega _{1} \times \frac{J _{1} - J _{2}}{J _{3}}) & 0
    \end{bmatrix}
\end{align*}

\begin{align*}
    B =
    \begin{bmatrix}
    \frac{1}{J _{1}} & 0 \\ 0 & 0 \\ 0 & \frac{1}{J _{3}}
    \end{bmatrix} \enspace
    C = 
        \begin{bmatrix}
            1 & 0 & 0 \\ 0 & 1 & 0 \\ 0 & 0 & 1
        \end{bmatrix} \enspace
    D =
        \begin{bmatrix}
            0 & 0 \\ 0 & 0 \\ 0 & 0
        \end{bmatrix}
\end{align*}
\end{document}
